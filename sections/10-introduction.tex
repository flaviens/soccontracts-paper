% !TeX root = ../main.tex
% !TeX spellcheck = en_US

Timing side-channel attacks exploit variations in execution time to leak secrets through measurable side effects.
By carefully choreographing benign-looking instruction sequences, an attacker can steer microarchitectural state, such as caches, TLBs, execution ports, branch predictors, or DRAM buffers, to make secret-dependent effects externally observable.
Two broad families have emerged: transient/speculative-execution attacks~\cite{kocher2019spectre,lipp2018meltdown,canella2019fallout,schwarz2019zombieload,van2019ridl,van2021cacheout,ragab2021rage,wikner2022retbleed,trujillo2023inception,wikner2023phantom,wikner2024breaking,wikner2025bpi}, which leave secret-dependent footprints during mis-speculated execution, and contention-driven channels on shared resources~\cite{bernstein2005cache,Liu2015LLC,YaromFalkner2014FlushReload,Yarom2016CacheBleed,Moghimi2018MemJam,Gruss2016PrefetchSCA,Pessl2016DRAMA}, which measure or induce interference to encode information.
These attacks are durable across processor generations because they can often be mounted entirely in software, without special privileges or hardware access.
Mitigation is notoriously difficult after fabrication, since leakage channels are entangled with performance-critical design choices.
Software or microcode defenses are often partial~\cite{ridlad} or impose significant overheads.

To address these challenges, recent work introduced hardware-software contracts, which inform developers about the side effects of their code on a contract-compliant CPU~\cite{guarnieri2021hardware,oleksenko2022revizor}.
As a result, several techniques have been proposed to formally verify that a CPU implementation complies with such contracts~\cite{dinesh2024conjunct,dinesh2025h,ceesay2024mucfi,wang2023specification,tan2025contractshadowlogic,hsiao2024rtl2mmupath} to prevent confidentiality breaches before chip fabrication.

\para{Platform timing channels}
CPUs are often verified in isolation~\cite{dinesh2024conjunct,dinesh2025h,ceesay2024mucfi,wang2023specification,tan2025contractshadowlogic,hsiao2024rtl2mmupath} before integration into a larger system, both for scalability and because at CPU design and verification time, the exact specifications of the integrating platforms are usually unknown during CPU design and verification.
However, hardware-software contracts inherently assume a specific system context. When verification ignores this context, the results may not hold once the CPU is integrated into a platform.
For example, the constant-time contract observer mode~\cite{guarnieri2021hardware} exposes the program counter and all memory-access addresses, capturing timing effects of caches and of the memory hierarchy~\cite{guarnieri2021hardware,oleksenko2022revizor}.
Other modes, such as the architectural observer, reveal even more detail by also disclosing values exchanged with memory.
Such observer modes depend on platform-level features (e.g., SoC components), yet many verification techniques overlook these modes or cannot flexibly capture platform-induced channels.

\para{Platform timing contracts}
Platform timing contracts extend the idea of hardware-software contracts to encompass the broader system context in which a CPU operates.
They aim to capture the timing behavior of the entire platform, including interactions with other components such as memory controllers, I/O devices, and interconnects.
By considering these interactions, platform timing contracts can provide a more comprehensive understanding of potential timing channels and their implications for security.
To verify the compliance of a CPU integrated into a specific platform using existing techniques that verify a CPU in isolation, we introduce an automatic instrumentation that captures the relevant platform-specific timing information while retaining compatibility with all existing hardware-software contract verification techniques.

We establish the platform timing contracts for four different platforms.
We apply the instrumentation corresponding to each platform to the four RISC-V CPUs: \textcolor{red}{TODO List them}.
Not only does the verification time increase by not more than ~\textcolor{red}{5}\%, but we demonstrate the existence of platform-specific timing channels on typical platforms, that would otherwise be missed.

In summary, our contributions in this paper are:
\begin{itemize}
    \item We introduce \pics that summarize platform-specific timing behaviors.
    \item We implement an open-source lightweight CPU instrumentation that accounts for all platform-induced timing side-channels.
    \item We apply instrumentations corresponding to various platforms to the 4 RISC-V CPUs to capture platform-specific timing information and show that these platforms can introduce unique timing channels that may not be present in isolated CPU verification and that would be missed by some existing verification techniques.
\end{itemize}

\vspace*{0.5em}
All our experiments can be found at: \url{https://placeholder/} 
