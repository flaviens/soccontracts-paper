Listing~\ref{lst:system-prompt-template} introduces the global prompt template.

\begin{lstlisting}[language=python,caption={Global prompt template},label={lst:system-prompt-template},breaklines=true]
"""You are an expert SystemVerilog designer. Your task is to generate a single SystemVerilog file containing MULTIPLE, executable SystemVerilog modules.
The goal is to create diverse snippets, each targeting as many lines as possible in the file ('{target_cpp_filename}'). You will be shown the content of this C++ file. The overall goal is to maximise the coverage of the cpp file.
CRITICAL REQUIREMENTS FOR EACH MODULE:
1.  Each module MUST have at least one input port (you can choose width/type) and at least one output port (you can choose width/type).
2. All class instanciations, must be done in a procedural blocks.
3. Each module of the file must be executable within the context of the file.
The generated SystemVerilog code MUST be syntactically correct and will be verified by Verilator.
Do NOT include:
- Any module named 'top'
- Simulation stimulus (like `initial` blocks driving signals over time or containing delays).
- Timing delays (`#`).
- Simulation control tasks (`$finish`, `$stop`, etc.).
- Display or monitoring tasks (`$display`, `$monitor`, `$write`, etc.).
- Module instantiations (unless the C++ file specifically handles hierarchy).
- Any comments or code that are not relevant to the SystemVerilog constructs.
The generated code should be self-contained and not rely on any external files or modules.
Provide only the complete SystemVerilog code containing all the generated modules in a single code block. Do not include explanations or markdown formatting."""
\end{lstlisting}

Listing~\ref{lst:initial-prompt-template} introduces the initial prompt template for each conversation.

\begin{lstlisting}[language=python,caption={initial prompt template},label={lst:initial-prompt-template},breaklines=true]
"""FIRST identify all the different SystemVerilog features this file is supposed to handle or simulate. THEN Once you identified all the features, that are specific to this file then generate a single SystemVerilog file containing MULTIPLE, modules.
Each module should target as many lines as possible from the Verilator C++ file '{target_cpp_filename}' (content provided below).
Analyze the C++ code to understand the Verilog constructs it handles. Generate complex but valid SystemVerilog snippets using these constructs, maximising the coverage of the C++ file.
CRITICAL REQUIREMENTS FOR EACH MODULE:
1.  Each module MUST have at least one input and one output.
2.  Each module must be executable within the file context.
3.  If you have no idea how to correct the error, delete the line that is problematic.
4.  Each module must have code to maximise the coverage of the C++ file.
Include as many DIFFERENT modules as possible, each focusing on a different feature this C++ file is supposed to handle.
Maximise the coverage of the C++ file.
Ensure the SystemVerilog code is self-contained (minimal/no instantiations) and syntactically correct.
Avoid simulation-specific constructs (initial blocks for stimulus, delays, $finish, $display).

The objective is MAXIMIZING the coverage of the C++ file.

Content of target C++ file '{target_cpp_filename}':
```cpp
{target_cpp_content}
```

Coverage of this file is '{coverage_excerpt}'.

As you can see it is quite low, and you must improve it by all means.

Create at least a module for each different function present in the file.

Generate only the SystemVerilog code containing all modules for this single file."""
\end{lstlisting}

Listing~\ref{lst:feedback-prompt-template} introduces the feedback prompt template, which is used in the feedback loop.

\begin{lstlisting}[language=python,caption={Feedback prompt template},label={lst:feedback-prompt-template},breaklines=true]
"""The following SystemVerilog code was generated to maximise coverage of the Verilator C++ file '{target_cpp_filename}', but it failed linting (`verilator --lint-only -Wall --no-timing`). The goal was to generate MULTIPLE modules, each targeting a specific construct, each module having an input and an output port.

Target C++ file content ('{target_cpp_filename}'):
```cpp
{target_cpp_content}
```

Coverage of this file is '{coverage_excerpt}'.

Faulty generated SystemVerilog code:
```systemverilog
{generated_v_code}
```

Verilator Lint Error Output (Stderr focused):
```text
{error_summary}
```

Please analyze the faulty SystemVerilog code, and the lint errors. Provide a corrected version of the SystemVerilog code that fixes the lint error(s) while adhering strictly to ALL the requirements:
1.  Each module targets a specific construct relevant to the C++ file, try not to simplify too much the code.
2.  EACH module MUST have at least one input and one output.
3.  Maximize the coverage of the C++ file.
4.  No comments
5.  FIX ALL THE ERRORS (try avoiding deleting the lines unless you have absolutely no idea how to fix the error).
Ensure the code remains self-contained within the single file and avoids simulation-specific constructs or timing delays.
Generate only the corrected SystemVerilog code containing all modules."""
\end{lstlisting}