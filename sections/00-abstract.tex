The time that a computer requires to perform computations can breach confidentiality.
Constant-time programming techniques defend against such timing attacks, but make assumptions on the behavior of the underlying hardware.
As a response, previous work has introduced several techniques for verifying CPU compliance with hardware-software contracts.
%
However, selecting which hardware-software contract to verify is non-trivial and requires careful consideration of the system's architecture and the specific timing attacks it may face: if a CPU is integrated into a platform that is subject to more timing channels than specified in the contract that the CPU expects, the verification may not be sufficient to guarantee security.
In this paper, we address the integration of a verified constant-time CPU into a larger system by introducing \pics.
\pics are a novel approach to specifying an upper bound of the timing channels that might appear in a the platform that integrates a CPU.
We introduce a lightweight instrumentation that allows for the dynamic monitoring of platform-induced timing channels and that is by construction compatible with all CPU constant-time verification techniques.
Based on this instrumentation, we show that several techniques, while effective in isolation, fail to provide security guarantees when the CPU is integrated into a larger system.
