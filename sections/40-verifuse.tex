\begin{figure*}[htp]

    \includegraphics[width=0.7\textwidth]{figures/verifuse_arch/verifuse-arch.drawio.pdf}

    \caption{Overview of \ourname.}
    \label{fig:verifuse-arch}
\end{figure*}

Figure~\ref{fig:verifuse-arch} illustrates the workflow of \ourname (Sections~\ref{sec:verifuse} and~\ref{sec:disparities}), preceded by the base snippet generator (Section~\ref{sec:codegen}).

\para{Base snippet generator}
Coverage data from EDA software source code is collected and used to prompt the LLM to produce SystemVerilog snippets aimed at improving coverage (\circled{1}).
Each snippet is refined interactively, up to ten times, or until it compiles without error in the target tool (\circled{2}).
Validated snippets are then differentially tested across multiple RTL simulators (\circled{3}).
If results diverge, the snippet is manually analyzed to determine whether the discrepancy stems from non-determinism or a simulator bug (\circled{4}).
Otherwise, it is added to the base pool for later use (\circled{5}).

\para{\ourname}
\ourname iteratively selects and parses base snippets (\circled{6}), combining them into more complex test cases (\circled{7}).
For each case, it generates both C++ and SystemVerilog testbenches (\circled{8}) and runs differential fuzzing for up to 1000 stimuli (\circled{9}).
Any mismatches are triaged for manual investigation (\circled{10}).
